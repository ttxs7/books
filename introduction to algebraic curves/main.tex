\documentclass{book}
\usepackage{amsmath}
\usepackage{amsfonts}
\usepackage{amssymb}
\usepackage{amsthm}
\usepackage{hyperref}
\def\R{\mathbb{R}}
\def\C{\mathbb{C}}
\def\max{\operatornamewithlimits{max}}
\def\Lip#1{\operatornamewithlimits{Lip}\left(#1\right)}
\newtheorem{theorem}{Theorem}
\newtheorem{proposition}{proposition}
\newtheorem{lemma}{Lemma}
\newtheorem{definition}{Definition}
\bibliographystyle{plain}
\title{Introduction to Algebraic Curves}
\author{TTXS}
\begin{document}
    \maketitle
    \tableofcontents
    \chapter{Fundamental Concepts}
    \section{Algebraic curves in the complex projective plane \texorpdfstring{$P^2\C$}{P2C}}
    Suppose $f(x,y)$ is a polynomial with real coefficients in two variables. Then the grapth in $\R^2$ which is defined by the equation
    \begin{equation}\label{e1.1.1}
        f(x,y)=0
    \end{equation}
    is called a \textit{real algebraic curve}, the degree of $f(x,y)$ is called the \textit{degree} of this algebraic curve. The reader is undoubtedly already familiar with algebraic curves of degree 1 and 2, which are the straight lines, ellipses, hyperbolas, and parabolas. However, if we are to discuss more general algebraic curves within the realm of real numbers, several problems arise which make it difficult to obtain complete results. The main reason for this is that the real field is not an algebraically closed field. For example, suppose we wish to discuss the number of intersection points of a straight line $L$ with the algebraic curve $C$ given by \eqref{e1.1.1}. Without any loss of generality, we assume that $L$ passes through the origin (this requirement can be easily satisfied by a suitable transformation of coordinates), so that its parametric equations can be written as: 
    \begin{equation}\label{e1.1.2}
        \begin{cases}
            x=\alpha t\\
            y=\beta t
        \end{cases}
    \end{equation}
    Moreover, suppose
    $$f(x,y)=f_n(x,y)+f_{n-1}(x,y)+\cdots+f_0, $$
    where each $f_k(x,y)$ is a homogeneus polynomial of degree $k$. By substituting \eqref{e1.1.2} into \eqref{e1.1.1}, we obtain
    \begin{equation}\label{e1.1.3}
        f_n(\alpha,\beta)t^n+f_{n-1}(\alpha,\beta)t^{n-1}+\cdots+f_0=0. 
    \end{equation}
    

\end{document}