\documentclass{book}
\usepackage{amsmath}
\usepackage{amsfonts}
\usepackage{amssymb}
\usepackage{amsthm}
\usepackage{hyperref}
\usepackage{authblk}
\newtheorem{theorem}{Theorem}
\newtheorem{proposition}{proposition}
\newtheorem{lemma}{Lemma}
\newtheorem{definition}{Definition}
\def\theoremautorefname{theorem}
\def\propositionautorefname{proposition}
\def\lemmaautorefname{lemma}
\def\definitionautorefname{definition}
\def\R{\mathbb{R}}
\def\C{\mathbb{C}}
\def\max{\operatornamewithlimits{max}}
\def\Lip#1{\operatornamewithlimits{Lip}\left(#1\right)}
\bibliographystyle{plain}
\title{Introduction to Algebraic Curves}
\author{Phillip A. Griffiths}
\author{Zhang Zhusheng}
\author{Zhao Chunlai}
\author{Zhou Qing}
\author{Kuniko Weltin}
\author{TTXS}
\affil{ttxs@pku.edu.cn}
\begin{document}
    \maketitle
    \tableofcontents
    \chapter{Fundamental Concepts}
    \section{Algebraic curves in the complex projective plane \texorpdfstring{$P^2\C$}{P2C}}
    Suppose $f(x,y)$ is a polynomial with real coefficients in two variables. Then the grapth in $\R^2$ which is defined by the equation 
    \begin{equation}\label{e1.1.1}
        f(x,y)=0
    \end{equation}
    is called a \textit{real algebraic curve}, the degree of $f(x,y)$ is called the \textit{degree} of this algebraic curve. The reader is undoubtedly already familiar with algebraic curves of degree 1 and 2, which are the straight lines, ellipses, hyperbolas, and parabolas. However, if we are to discuss more general algebraic curves within the realm of real numbers, several problems arise which make it difficult to obtain complete results. The main reason for this is that the real field is not an algebraically closed field. For example, suppose we wish to discuss the number of intersection points of a straight line $L$ with the algebraic curve $C$ given by \eqref{e1.1.1}. Without any loss of generality, we assume that $L$ passes through the origin (this requirement can be easily satisfied by a suitable transformation of coordinates), so that its parametric equations can be written as: 
    \begin{equation}\label{e1.1.2}
        \begin{cases}
            x=\alpha t\\
            y=\beta t
        \end{cases}
    \end{equation}
    Moreover, suppose 
    $$f(x,y)=f_n(x,y)+f_{n-1}(x,y)+\cdots+f_0, $$
    where each $f_k(x,y)$ is a homogeneus polynomial of degree $k$. By substituting \eqref{e1.1.2} into \eqref{e1.1.1}, we obtain 
    \begin{equation}\label{e1.1.3}
        f_n(\alpha,\beta)t^n+f_{n-1}(\alpha,\beta)t^{n-1}+\cdots+f_0=0. 
    \end{equation}
    To determine the number of roots of this equation over $\R$ is by no means a simple matter. Moreover, we cannot expect a neat conclusion since the number of real roots is dependent on the nature of the coefficients. But when we consider $f(x,y)$ as a polynomial with complex coefficients in two variables and consider \eqref{e1.1.1} as an algebraic curve in $\C^2$, then the number of intersection points of the complex straight line \eqref{e1.1.2} with the complex algebraic curve \eqref{e1.1.1} is again given by equation \eqref{e1.1.3}. From the well-known fundamental theoerm of algebra, so long as $f_n(\alpha,\beta)\neq0$, equation \eqref{e1.1.3} will have exactly $n$ roots (multiple roots counted with multiplicity). In other words, the complex algebraic curve $C$ of degree $n$ and the complex line $L$ intersect at $n$ points (the intersection points corresponding to multiple roots are thought of as multiple points of intersection). There is one exception to the preceding discussion: if 
    \begin{gather*}
        f_n(\alpha,\beta)=f_{n-1}(\alpha,\beta)=\cdots=f_{m+1}(\alpha,\beta)=0, \\
        f_m(\alpha,\beta)\neq0, 
    \end{gather*}
    then $L$ and $C$ intersect in only $m$ points in $\C^2$. In this case we may regard the remaining $n-m$ points of intersection as being at infinity. This viewpoint can be explained in more detail as follows: substituting $1/s$ for $t$ in \eqref{e1.1.3} and multiplicity both sides of the equation by $s^n$, we have 
    \begin{equation}\label{e1.1.3'}\tag{\ref*{e1.1.3}'}
        f_n(\alpha,\beta)+f_{n-1}(\alpha,\beta)s+\cdots+f_0(\alpha,\beta)s^n=0. 
    \end{equation}
    If $f_n(\alpha,\beta)\neq0$, then $s=0$ (which corresponding to $t=\infty$) is not a root of \eqref{e1.1.3'}; if 

\end{document}