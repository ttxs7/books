\chapter{基本概念}\label{c1}
\section{代数曲线}\label{s1-1}
\begin{definition}
    在$\C^3\backslash\{(0,0,0)\}$上建立等价关系: $(\zeta,\xi,\eta)\sim(\zeta',\xi',\eta')$当且仅当$\exists\lambda\in\C$, $\lambda\neq0$使得$(\zeta,\xi,\eta)=\lambda(\zeta',\xi',\eta')$. 定义$[\zeta,\xi,\eta]$是$(\zeta,\xi,\eta)$所在的等价类, \textit{复射影平面}$\P^2=(\C^3\backslash\{(0,0,0)\})/\sim$. 
\end{definition}
\begin{definition}
    设$F(\zeta,\xi,\eta)$是齐次多项式, 称$C=\{[\zeta,\xi,\eta]|F(\zeta,\xi,\eta)=0\}$为$\deg F$次\textit{代数曲线}. 如果$F$不可约, 称$C$不可约. 
\end{definition}
\begin{remark}$F(\zeta,\xi,\eta)=\zeta^{\deg F}F(1,\frac\xi\zeta,\frac\eta\zeta)$, 从而代数曲线由它在$\C$上的投影和次数决定. 
\end{remark}
\begin{definition}
    设$F_i$是不可约齐次多项式, $C_i$是$F_i$决定的代数曲线, 定义$m_1C_1+\cdots+m_l C_l$是$F_1^{m_1}\cdots F_l^{m_l}$决定的代数曲线. 
\end{definition}

\section{Riemann面}\label{s1-2}
\begin{definition}
    设$C$是连通Hausdorff空间, $\{U_\alpha\}$是$C$的开覆盖, $z_\alpha:U_\alpha\to\C$是到开集$z_\alpha(U_\alpha)$的同胚映射, $z_\beta\circ z_\alpha^{-1}$是$z_\alpha(U_\alpha\cap U_\beta)$上的全纯单射, 称$C$为\textit{Riemann面}, $(U_\alpha,z_\alpha)$为\textit{全纯坐标卡}, $\{(U_\alpha,z_\alpha)\}$为\textit{全纯坐标卡系}. 
\end{definition}
\begin{definition}
    设复数$w_1,w_2$实线性无关, 称\\
    $\Lambda=\{m_1w_1+m_2w_2|m_1,m_2\in\Z\}$为\textit{格}, 
    $\C/\Lambda$为\textit{复环面}. 显然$\C/\Lambda$是紧Riemann面. 
\end{definition}
\begin{definition}
    设$C$是紧Riemann面, 显然$C$是紧致光滑可定向的实二维流形, 从而同胚于一个有若干环柄的球面, 定义$C$的\textit{亏格}为环柄的个数, \textit{Euler示性数}为$\chi=2-2g$. 
\end{definition}
\begin{definition}
    设$C$是不可约代数曲线, $\widetilde{C}$是紧Riemann面, $\sigma:\widetilde{C}\to\P^2$是全纯映射满足$\sigma(\widetilde{C})=C$且在$C$的光滑点上一一, 称$(\widetilde{C},\sigma)$为$C$的\textit{正则化}. 
\end{definition}
\begin{theorem}
    设$C$是不可约代数曲线, 则存在$C$的正则化. 
\end{theorem}
\begin{theorem}
    设$\widetilde{C}$是紧Riemann面, 则存在至多具有通常二重点的代数曲线$C$和映射$\sigma$使得$(\widetilde{C},\sigma)$为$C$的正则化. 
\end{theorem}

\section{全纯与半纯函数}\label{s1-3}
\begin{definition}
    设$C$是Riemann面, 
\end{definition}
\section{全纯与半纯微分}\label{s1-4}

\section{微分形式}\label{s1-5}

\section{Poincar\texorpdfstring{\'e}{e}-Hopf公式}\label{s1-6}

\section{复流形}\label{s1-7}

\section{代数簇}\label{s1-8}

\section{光滑点, 切空间, 隐函数定理}\label{s1-9}

\section{紧Riemann面到复射影空间的全纯映射}\label{s1-15}
