\chapter{正则化定理及其应用}\label{c2}
\section{代数曲线的奇点}\label{s2-1}

\section{不可约代数曲线的连通性}\label{s2-2}

\section{Weierstrass多项式}

\section{代数曲线的局部构造}
\begin{lemma}
    设$C$是代数曲线, $p\in C$. 可以选坐标系使得$p=[1,0,0]$, 且$C$的仿射多项式等于$y^n+a_1(x)y^{n-1}+\cdots+a_n(x)$, 其中$\deg a_i\leq i$. 
\end{lemma}


\section{正则化定理}

\section{除子, 相交数}
\begin{definition}
    设$\widetilde{C}$是Riemann面. 称$D=m_1p_1+\cdots+m_lp_l$是$\widetilde{C}$的\textit{除子}, 其中$m_i\in\Z$, $p_i\in\widetilde{C}$. 定义$D$的\textit{次数}$\deg(D)=\sum m_i$. 定义$\Div(\widetilde{C})$为$\widetilde{C}$的除子集. 显然$\Div(\widetilde{C})$是Abel群. 
\end{definition}
\begin{definition}
    设$\widetilde{C}$是紧Riemann面, $f\in K(\widetilde{C})$. 定义$(f)=\sum\nu_p(f)p$. 显然$(f)\in\Div(\widetilde{C})$, $\deg(f)=0$. 
\end{definition}
\begin{definition}
    
\end{definition}

\section{Riemann-Hurwitz公式}

\section{亏格公式}
